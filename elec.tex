% 投稿文章的 TeX 模板(含引用部分)
\documentclass[12pt]{article}
\usepackage{graphicx}
\usepackage{amsmath,amssymb}
\usepackage{hyperref}
\usepackage[utf8]{inputenc}
\usepackage{geometry}
\geometry{a4paper, margin=1in}
\usepackage{cite} % 管理引用

\title{ZheDa DataSet in China}
\author{Name1, Name2, Name3}
\date{\today}

\begin{document}
\maketitle

\begin{abstract}
% 在此处填写摘要
\end{abstract}

\section*{Background \& Summary}
% 请在此处添加内容。例如:文献\cite{example1}中介绍了相关背景。

\section*{Methods}
\subsection*{Data collection}
% 数据采集内容

\subsection*{Data processing}
% 数据处理内容
Some data files are missing due to failures in data collection or processing, leading to unavailability of certain fields (e.g., \(\mathrm{PQ\_nodes}, \mathrm{PV\_nodes}, \mathrm{voltage\_profile}, \mathrm{edges}\)). To address this, the method scans remaining valid files and uses their timestamps to locate adjacent files in time, ensuring that similar characteristics can be gathered to fill these missing entries.

The primary data imputation relies on a ``voting'' or majority-decision approach. First, the code identifies valid files situated around the missing timestamp (for example, one day before and after), while skipping any files that are also in the missing set. Next, it consolidates the boolean arrays from these valid files by stacking them. For a given index, if more than half of the stacked entries are \texttt{True}, the estimated value at that index is set to \texttt{True}; otherwise, it becomes \texttt{False}. For handling the \(\mathrm{edges}\) field, the code takes the union of all edges collected from these adjacent files and then converts this union into a sorted array. By putting these steps together, the approach effectively reconstructs the missing data using the existing information from nearby time segments.

In practice, for each missing file, the function \texttt{get\_adjacent\_data()} locates usable data within a set time window, such as one day around the target timestamp, ignoring other timestamps that are also missing. These data are stacked for the boolean fields (i.e., \(\mathrm{PQ\_nodes}, \mathrm{PV\_nodes}, \mathrm{voltage\_profile}\)) in order to apply the majority vote for each index. Meanwhile, a union operation is performed on the sets of edges from those adjacent files to handle the \(\mathrm{edges}\) field. In the end, the code creates a new \(\mathrm{npz}\) file containing the reconstructed fields for the missing data and saves it to a specified directory.

One key advantage of this method is its simplicity: it does not require training sophisticated models or tuning complex parameters, as the data are gathered and used directly from adjacent segments. Another strength lies in its stability. When adjacent periods have similar patterns, majority voting can robustly capture the common state without overcomplicating the process. Finally, it boasts strong versatility because the procedure requires minimal assumptions or prior knowledge, making it suitable for many scenarios where partial data are still available for the time windows near missing entries.



\section*{Data Records}
% 数据记录说明
11
\section*{Technical Validation}
\subsection*{Missing value imputation}
% 缺失值填补说明

\subsection*{Fault value diagnosis}
% 故障值诊断说明

\subsection*{Correlation analysis}
% 相关性分析说明
111\cite{example1}
\subsection*{Load curve}
% 负荷曲线说明

\section*{Usage Notes}
% 使用说明

\section*{Acknowledgements}
% 致谢内容

\section*{References}
% 请在此处插入引用,可以采用 BibTeX 或手动添加参考文献。
\bibliographystyle{unsrt}  % 可选 plain、alpha、abbrv 等格式
\bibliography{references}  % 请确保有一个名为 references.bib 的 BibTeX 数据库文件


\end{document}
