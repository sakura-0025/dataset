\documentclass[12pt]{article}
\usepackage{graphicx}
\usepackage{amsmath,amssymb}
\usepackage{hyperref}
\usepackage[utf8]{inputenc}
\usepackage{geometry}
\geometry{a4paper, margin=1in}
\usepackage{cite}

\title{ZheDa DataSet in China}
\author{Name1, Name2, Name3}
\date{\today}

\begin{document}
\maketitle

\begin{abstract}
    
\end{abstract}

\section*{Background \& Summary}

The stability and operational efficiency of power systems are crucial for modern energy infrastructure,
especially in the context of large-scale renewable energy integration and the continuous development of smart grid technologies.
Understanding the dynamic behavior of power systems, including interactions between different types of nodes, is essential for grid optimization,
stability analysis, and operational scheduling. However, publicly available datasets that comprehensively reflect the evolving topology and electrical
characteristics of power networks over time remain limited. To address this gap, we have constructed a time-series dataset that records the topological
relationships and electrical properties of power system nodes, supporting research in power flow analysis, anomaly detection, and grid resilience assessment.  

This dataset has the following key characteristics:

\begin{itemize}
    \item \textbf{\small Topological Structure Modeling}: The \texttt{edges} array records the connections between power system nodes, providing a clear representation of the grid topology. This facilitates complex network analysis, fault propagation studies, and network optimization.  
    \item \textbf{\small Time-Series Electrical Characteristics}: The \texttt{H} matrix contains electrical parameters (e.g., voltage, current, power) for each node at different time points, supporting load analysis, power flow calculations, and time-series studies.  
    \item \textbf{\small Node Function Classification}: Three types of nodes are identified within the dataset:
    \begin{itemize}
        \item \textbf{\small PQ Nodes (Load Nodes)}: Nodes with both active power (P) and reactive power (Q) demands, which are essential for studying power distribution and dynamic load variations.  
        \item \textbf{\small PV Nodes (Generation Nodes)}: Nodes that provide active power output while maintaining a constant voltage, useful for analyzing generator operations and optimization strategies.  
        \item \textbf{\small Vt Nodes (Voltage-Controlled Nodes)}: Typically representing critical substations or power plants responsible for voltage stability, these nodes are valuable for studying grid regulation and control strategies.  
    \end{itemize}
    \item \textbf{\small Temporal Data Features}: The inclusion of time-series data enables dynamic grid analysis, such as short-term load forecasting, stability assessment, and anomaly detection, rather than being limited to static power flow calculations.  
    \item \textbf{\small Wide Applicability}: The dataset can be used in machine learning, deep learning, optimization algorithm testing, power grid fault prediction, power quality analysis, and smart grid control strategy research.  
\end{itemize}

The time-series nature of this dataset makes it suitable for a wide range of research applications,
including machine learning-based power system forecasting, stability assessment, and anomaly detection.
Additionally, it serves as a benchmark for optimization and control algorithm testing, contributing to the development of intelligent scheduling methods.
Unlike traditional static datasets, which typically provide only snapshots of power system states at specific moments, this dataset captures the dynamic evolution of
the grid, allowing researchers to analyze both long-term trends and short-term fluctuations. The improved temporal resolution enhances its applicability to real-world
power system analysis.  

Several datasets, such as IEEE test systems and open-source grid models, have been widely used in power system research. 
However, these datasets primarily focus on steady-state analysis and lack representations of dynamic system behavior. Additionally, 
while frameworks like MATPOWER provide computational tools for power flow analysis, they do not inherently include real-world time-series data for model validation.
 In contrast, our dataset fills this gap by offering structured time-series data, expanding the possibilities for computational experiments and analytical studies.  

This dataset holds significant reuse potential across multiple domains:

\begin{itemize}
    \item \textbf{\small Power Grid Stability and Security Analysis}: Supports the assessment of grid operational states and stability under varying load conditions and fault scenarios.  
    \item \textbf{\small Smart Grid Optimization and Scheduling}: Enables research on real-time grid optimization and scheduling, improving energy efficiency and reducing operational costs.  
    \item \textbf{\small Data-Driven Predictive Models}: Suitable for training and validating machine learning and deep learning models for tasks such as load forecasting and anomaly detection.  
    \item \textbf{\small Interdisciplinary Research}: Can be integrated with economic and market factors to study the financial impact of grid anomalies and optimize energy management strategies for economic benefits.  
\end{itemize}

Overall, this dataset provides a comprehensive, high-resolution representation of power system node interactions,
creating new research opportunities in grid modeling, machine learning applications, and energy system optimization.
By making this dataset publicly available, we aim to promote reproducible research and accelerate advancements in power system management and analysis.

\section*{Methods}
\subsection*{Data collection}
% 数据采集内容

\subsection*{Data processing}
% 数据处理内容

\section*{Data Records}
% 数据记录说明

\section*{Technical Validation}
\subsection*{Missing value imputation}
% 缺失值填补说明

\subsection*{Fault value diagnosis}
% 故障值诊断说明

\subsection*{Correlation analysis}
% 相关性分析说明
\subsection*{Load curve}
% 负荷曲线说明

\section*{Usage Notes}
% 使用说明

\section*{Acknowledgements}
% 致谢内容

% 请在此处插入引用,可以采用 BibTeX 或手动添加参考文献。
\bibliographystyle{unsrt}  % 可选 plain、alpha、abbrv 等格式
\bibliography{references}  % 请确保有一个名为 references.bib 的 BibTeX 数据库文件


\end{document}
